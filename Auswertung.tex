\section{Auswertung}
Zur Berechnung des Wasserwertes, sind die Temperaturdifferenzen über die Zeit in Abbildungen \ref{abb:benz1} und \ref{abb:Benz2} zu sehen.
Es wurde eine Lineare Regressionsgerade über die ersten 6 Werte berechnet. 
Eine Zweite wurde über die Letzen vier Werte gelegt.
Über Geogebra wurde die Fläche zwischen einer Vertikalen, den Messpunkten und den Ausgleichsgeraden berechnet.
Die Vertikale wurde so in die Abbildung gelegt, das beide Flächen gleich groß sind.
Die Schnittpunkte der Vertikalen mit den Ausgleichgeraden, gibt die Temperaturdifferenz $\Delta T_\alpha$.
Diese ist befreit von einigen Fehlern, die aus der Wärmestrahlung des Gerätes, sowie der Temperaturdifferenz zwischen Wasser und Bombe, entsteht.
In Tabelle \ref{tab:1w} sind die berechneten Ausgleichsgeraden, die Vertikalen, Probengewichte, sowie die Temperaturdifferenzen dargestellt.
Gegeben wurden uns die spezifische Verbrennungsenthalpie von Benzoesäure mit \qty{26439}{\joule\per\gram}.
Und der Verbrennungenthalpie der Baumwollfäden mit \qty{50}{\joule}.
Daraus ergibt sich nach Gleichung \ref{eq:ww} der Wasserwert.
Beispielhaft ist dies für die erste Benzoesäure-Tablette, mit \qty{0,4962}{\gram}, in Gleichung \ref{eqbsp:ww} berechnet


Dies wurde analog für die Zweite Tablette durchgeführt.
Die Temperaturdifferenzen der Proben sind in den Abbildungen \ref{abb:gummi1} und \ref{abb:gummi2} dargestellt.
Deren $\Delta T_\alpha$ kann nun wieder über Gleichung \ref{eq:ww} in eine Wärmeenergie umgerechnet werden.
Die Verbrennungswärme geteilt durch die Probenmasse gibt nun die Spezifsche Verbrennungswärme. 
Da in unserem Versuch die Druckänderung nicht Null ist, kann die Spezifische verbrennungswärme nicht ohne weiteres mit der verbrennungsenthalpie gleichgesetzt werden.
Allerdings liegen uns keine Werte über die Druckänderung vor, so definieren wir den Druck als konstant.
Die erhaltenen  Ergebnisse können aus der Tabelle \ref{tab:erg} entnommen werden.
Der Fehler der Wasserwerte ist zurückzuführen auf die Messunsicherheit, der Waage und somit nach Gleichung \ref{eq:dW} berechnet worden.
Beispielhaft ist dies für die erste Benzoesäure-Tablette in Gleichung \ref{eqbsp:dW} gezeigt.
Weiter ergiebt sich der Fehler des Spezifischen Verbrennungsenthalpie nach den Gleichungen \ref{eq:dQ} und \ref{eqbsp:dQ}.
