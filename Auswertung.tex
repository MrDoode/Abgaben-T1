\section{Auswertung}
\begin{figure}[t]
	\centering
	\begin{subfigure}{0.4\textwidth}
		\centering
		\begin{tikzpicture}[gnuplot, scale=0.5, every node/.style={scale=0.5}]
%% generated with GNUPLOT 5.4p5 (Lua 5.4; terminal rev. Jun 2020, script rev. 115)
%% Do 24 Nov 2022 22:18:53 CET
\path (0.000,0.000) rectangle (12.500,8.750);
\gpcolor{color=gp lt color border}
\gpsetlinetype{gp lt border}
\gpsetdashtype{gp dt solid}
\gpsetlinewidth{1.00}
\draw[gp path] (1.320,0.985)--(1.500,0.985);
\draw[gp path] (11.947,0.985)--(11.767,0.985);
\node[gp node right] at (1.136,0.985) {$0$};
\draw[gp path] (1.320,1.917)--(1.500,1.917);
\draw[gp path] (11.947,1.917)--(11.767,1.917);
\node[gp node right] at (1.136,1.917) {$0.2$};
\draw[gp path] (1.320,2.849)--(1.500,2.849);
\draw[gp path] (11.947,2.849)--(11.767,2.849);
\node[gp node right] at (1.136,2.849) {$0.4$};
\draw[gp path] (1.320,3.781)--(1.500,3.781);
\draw[gp path] (11.947,3.781)--(11.767,3.781);
\node[gp node right] at (1.136,3.781) {$0.6$};
\draw[gp path] (1.320,4.713)--(1.500,4.713);
\draw[gp path] (11.947,4.713)--(11.767,4.713);
\node[gp node right] at (1.136,4.713) {$0.8$};
\draw[gp path] (1.320,5.645)--(1.500,5.645);
\draw[gp path] (11.947,5.645)--(11.767,5.645);
\node[gp node right] at (1.136,5.645) {$1$};
\draw[gp path] (1.320,6.577)--(1.500,6.577);
\draw[gp path] (11.947,6.577)--(11.767,6.577);
\node[gp node right] at (1.136,6.577) {$1.2$};
\draw[gp path] (1.320,7.509)--(1.500,7.509);
\draw[gp path] (11.947,7.509)--(11.767,7.509);
\node[gp node right] at (1.136,7.509) {$1.4$};
\draw[gp path] (1.320,8.441)--(1.500,8.441);
\draw[gp path] (11.947,8.441)--(11.767,8.441);
\node[gp node right] at (1.136,8.441) {$1.6$};
\draw[gp path] (1.320,0.985)--(1.320,1.165);
\draw[gp path] (1.320,8.441)--(1.320,8.261);
\node[gp node center] at (1.320,0.677) {$0$};
\draw[gp path] (2.501,0.985)--(2.501,1.165);
\draw[gp path] (2.501,8.441)--(2.501,8.261);
\node[gp node center] at (2.501,0.677) {$100$};
\draw[gp path] (3.682,0.985)--(3.682,1.165);
\draw[gp path] (3.682,8.441)--(3.682,8.261);
\node[gp node center] at (3.682,0.677) {$200$};
\draw[gp path] (4.862,0.985)--(4.862,1.165);
\draw[gp path] (4.862,8.441)--(4.862,8.261);
\node[gp node center] at (4.862,0.677) {$300$};
\draw[gp path] (6.043,0.985)--(6.043,1.165);
\draw[gp path] (6.043,8.441)--(6.043,8.261);
\node[gp node center] at (6.043,0.677) {$400$};
\draw[gp path] (7.224,0.985)--(7.224,1.165);
\draw[gp path] (7.224,8.441)--(7.224,8.261);
\node[gp node center] at (7.224,0.677) {$500$};
\draw[gp path] (8.405,0.985)--(8.405,1.165);
\draw[gp path] (8.405,8.441)--(8.405,8.261);
\node[gp node center] at (8.405,0.677) {$600$};
\draw[gp path] (9.585,0.985)--(9.585,1.165);
\draw[gp path] (9.585,8.441)--(9.585,8.261);
\node[gp node center] at (9.585,0.677) {$700$};
\draw[gp path] (10.766,0.985)--(10.766,1.165);
\draw[gp path] (10.766,8.441)--(10.766,8.261);
\node[gp node center] at (10.766,0.677) {$800$};
\draw[gp path] (11.947,0.985)--(11.947,1.165);
\draw[gp path] (11.947,8.441)--(11.947,8.261);
\node[gp node center] at (11.947,0.677) {$900$};
\draw[gp path] (1.320,8.441)--(1.320,0.985)--(11.947,0.985)--(11.947,8.441)--cycle;
\draw[gp path](6.490,0.986)--(6.490,8.442);
\node[gp node center,rotate=-270] at (0.292,4.713) {$\Delta T$ / K};
\node[gp node center] at (6.633,0.215) {t / s};
\node[gp node right] at (2.424,5.021) {Daten};
\gpcolor{rgb color={0.580,0.000,0.827}}
\gpsetpointsize{4.00}
\gp3point{gp mark 1}{}{(2.028,1.321)}
\gp3point{gp mark 1}{}{(2.737,1.419)}
\gp3point{gp mark 1}{}{(3.445,1.461)}
\gp3point{gp mark 1}{}{(4.154,1.489)}
\gp3point{gp mark 1}{}{(4.862,1.512)}
\gp3point{gp mark 1}{}{(5.571,1.533)}
\gp3point{gp mark 1}{}{(6.279,4.078)}
\gp3point{gp mark 1}{}{(6.988,6.741)}
\gp3point{gp mark 1}{}{(7.696,7.330)}
\gp3point{gp mark 1}{}{(8.405,7.533)}
\gp3point{gp mark 1}{}{(9.113,7.611)}
\gp3point{gp mark 1}{}{(9.822,7.643)}
\gp3point{gp mark 1}{}{(10.530,7.656)}
\gp3point{gp mark 1}{}{(11.239,7.662)}
\gp3point{gp mark 1}{}{(11.947,7.660)}
\gp3point{gp mark 1}{}{(3.066,5.021)}
\gpcolor{color=gp lt color border}
\node[gp node right] at (2.424,4.713) {T1};
\gpcolor{rgb color={0.000,0.620,0.451}}
\draw[gp path] (2.608,4.713)--(3.524,4.713);
\draw[gp path] (2.028,1.399)--(2.129,1.403)--(2.229,1.407)--(2.329,1.411)--(2.429,1.415)%
  --(2.529,1.419)--(2.630,1.423)--(2.730,1.426)--(2.830,1.430)--(2.930,1.434)--(3.030,1.438)%
  --(3.131,1.442)--(3.231,1.446)--(3.331,1.450)--(3.431,1.454)--(3.531,1.458)--(3.631,1.462)%
  --(3.732,1.465)--(3.832,1.469)--(3.932,1.473)--(4.032,1.477)--(4.132,1.481)--(4.233,1.485)%
  --(4.333,1.489)--(4.433,1.493)--(4.533,1.497)--(4.633,1.501)--(4.734,1.505)--(4.834,1.508)%
  --(4.934,1.512)--(5.034,1.516)--(5.134,1.520)--(5.234,1.524)--(5.335,1.528)--(5.435,1.532)%
  --(5.535,1.536)--(5.635,1.540)--(5.735,1.544)--(5.836,1.548)--(5.936,1.551)--(6.036,1.555)%
  --(6.136,1.559)--(6.236,1.563)--(6.337,1.567)--(6.437,1.571)--(6.537,1.575)--(6.637,1.579)%
  --(6.737,1.583)--(6.837,1.587)--(6.938,1.590)--(7.038,1.594)--(7.138,1.598)--(7.238,1.602)%
  --(7.338,1.606)--(7.439,1.610)--(7.539,1.614)--(7.639,1.618)--(7.739,1.622)--(7.839,1.626)%
  --(7.940,1.630)--(8.040,1.633)--(8.140,1.637)--(8.240,1.641)--(8.340,1.645)--(8.440,1.649)%
  --(8.541,1.653)--(8.641,1.657)--(8.741,1.661)--(8.841,1.665)--(8.941,1.669)--(9.042,1.673)%
  --(9.142,1.676)--(9.242,1.680)--(9.342,1.684)--(9.442,1.688)--(9.543,1.692)--(9.643,1.696)%
  --(9.743,1.700)--(9.843,1.704)--(9.943,1.708)--(10.043,1.712)--(10.144,1.716)--(10.244,1.719)%
  --(10.344,1.723)--(10.444,1.727)--(10.544,1.731)--(10.645,1.735)--(10.745,1.739)--(10.845,1.743)%
  --(10.945,1.747)--(11.045,1.751)--(11.146,1.755)--(11.246,1.758)--(11.346,1.762)--(11.446,1.766)%
  --(11.546,1.770)--(11.646,1.774)--(11.747,1.778)--(11.847,1.782)--(11.947,1.786);
\gpcolor{color=gp lt color border}
\node[gp node right] at (2.424,4.405) {T2};
\gpcolor{rgb color={0.337,0.706,0.914}}
\draw[gp path] (2.608,4.405)--(3.524,4.405);
\draw[gp path] (2.028,7.584)--(2.129,7.585)--(2.229,7.585)--(2.329,7.586)--(2.429,7.587)%
  --(2.529,7.588)--(2.630,7.589)--(2.730,7.589)--(2.830,7.590)--(2.930,7.591)--(3.030,7.592)%
  --(3.131,7.593)--(3.231,7.594)--(3.331,7.594)--(3.431,7.595)--(3.531,7.596)--(3.631,7.597)%
  --(3.732,7.598)--(3.832,7.598)--(3.932,7.599)--(4.032,7.600)--(4.132,7.601)--(4.233,7.602)%
  --(4.333,7.602)--(4.433,7.603)--(4.533,7.604)--(4.633,7.605)--(4.734,7.606)--(4.834,7.606)%
  --(4.934,7.607)--(5.034,7.608)--(5.134,7.609)--(5.234,7.610)--(5.335,7.611)--(5.435,7.611)%
  --(5.535,7.612)--(5.635,7.613)--(5.735,7.614)--(5.836,7.615)--(5.936,7.615)--(6.036,7.616)%
  --(6.136,7.617)--(6.236,7.618)--(6.337,7.619)--(6.437,7.619)--(6.537,7.620)--(6.637,7.621)%
  --(6.737,7.622)--(6.837,7.623)--(6.938,7.624)--(7.038,7.624)--(7.138,7.625)--(7.238,7.626)%
  --(7.338,7.627)--(7.439,7.628)--(7.539,7.628)--(7.639,7.629)--(7.739,7.630)--(7.839,7.631)%
  --(7.940,7.632)--(8.040,7.632)--(8.140,7.633)--(8.240,7.634)--(8.340,7.635)--(8.440,7.636)%
  --(8.541,7.636)--(8.641,7.637)--(8.741,7.638)--(8.841,7.639)--(8.941,7.640)--(9.042,7.641)%
  --(9.142,7.641)--(9.242,7.642)--(9.342,7.643)--(9.442,7.644)--(9.543,7.645)--(9.643,7.645)%
  --(9.743,7.646)--(9.843,7.647)--(9.943,7.648)--(10.043,7.649)--(10.144,7.649)--(10.244,7.650)%
  --(10.344,7.651)--(10.444,7.652)--(10.544,7.653)--(10.645,7.654)--(10.745,7.654)--(10.845,7.655)%
  --(10.945,7.656)--(11.045,7.657)--(11.146,7.658)--(11.246,7.658)--(11.346,7.659)--(11.446,7.660)%
  --(11.546,7.661)--(11.646,7.662)--(11.747,7.662)--(11.847,7.663)--(11.947,7.664);
\gpcolor{color=gp lt color border}
\draw[gp path] (1.320,8.441)--(1.320,0.985)--(11.947,0.985)--(11.947,8.441)--cycle;
%% coordinates of the plot area
\gpdefrectangularnode{gp plot 1}{\pgfpoint{1.320cm}{0.985cm}}{\pgfpoint{11.947cm}{8.441cm}}
\end{tikzpicture}
%% gnuplot variables

		\caption{Temperaturdifferenzen der ersten Benzoesäuremessung}
		\label{abb:benz1}
	\end{subfigure}
	\hspace{5mm}
	\begin{subfigure}{0.4\textwidth}
		\centering
		\begin{tikzpicture}[gnuplot, scale=0.5, every node/.style={scale=0.5}]
%% generated with GNUPLOT 5.4p5 (Lua 5.4; terminal rev. Jun 2020, script rev. 115)
%% Do 24 Nov 2022 22:18:53 CET
\path (0.000,0.000) rectangle (12.500,8.750);
\gpcolor{color=gp lt color border}
\gpsetlinetype{gp lt border}
\gpsetdashtype{gp dt solid}
\gpsetlinewidth{1.00}
\draw[gp path] (1.320,0.985)--(1.500,0.985);
\draw[gp path] (11.947,0.985)--(11.767,0.985);
\node[gp node right] at (1.136,0.985) {$0$};
\draw[gp path] (1.320,1.917)--(1.500,1.917);
\draw[gp path] (11.947,1.917)--(11.767,1.917);
\node[gp node right] at (1.136,1.917) {$0.2$};
\draw[gp path] (1.320,2.849)--(1.500,2.849);
\draw[gp path] (11.947,2.849)--(11.767,2.849);
\node[gp node right] at (1.136,2.849) {$0.4$};
\draw[gp path] (1.320,3.781)--(1.500,3.781);
\draw[gp path] (11.947,3.781)--(11.767,3.781);
\node[gp node right] at (1.136,3.781) {$0.6$};
\draw[gp path] (1.320,4.713)--(1.500,4.713);
\draw[gp path] (11.947,4.713)--(11.767,4.713);
\node[gp node right] at (1.136,4.713) {$0.8$};
\draw[gp path] (1.320,5.645)--(1.500,5.645);
\draw[gp path] (11.947,5.645)--(11.767,5.645);
\node[gp node right] at (1.136,5.645) {$1$};
\draw[gp path] (1.320,6.577)--(1.500,6.577);
\draw[gp path] (11.947,6.577)--(11.767,6.577);
\node[gp node right] at (1.136,6.577) {$1.2$};
\draw[gp path] (1.320,7.509)--(1.500,7.509);
\draw[gp path] (11.947,7.509)--(11.767,7.509);
\node[gp node right] at (1.136,7.509) {$1.4$};
\draw[gp path] (1.320,8.441)--(1.500,8.441);
\draw[gp path] (11.947,8.441)--(11.767,8.441);
\node[gp node right] at (1.136,8.441) {$1.6$};
\draw[gp path] (1.320,0.985)--(1.320,1.165);
\draw[gp path] (1.320,8.441)--(1.320,8.261);
\node[gp node center] at (1.320,0.677) {$0$};
\draw[gp path] (2.501,0.985)--(2.501,1.165);
\draw[gp path] (2.501,8.441)--(2.501,8.261);
\node[gp node center] at (2.501,0.677) {$100$};
\draw[gp path] (3.682,0.985)--(3.682,1.165);
\draw[gp path] (3.682,8.441)--(3.682,8.261);
\node[gp node center] at (3.682,0.677) {$200$};
\draw[gp path] (4.862,0.985)--(4.862,1.165);
\draw[gp path] (4.862,8.441)--(4.862,8.261);
\node[gp node center] at (4.862,0.677) {$300$};
\draw[gp path] (6.043,0.985)--(6.043,1.165);
\draw[gp path] (6.043,8.441)--(6.043,8.261);
\node[gp node center] at (6.043,0.677) {$400$};
\draw[gp path] (7.224,0.985)--(7.224,1.165);
\draw[gp path] (7.224,8.441)--(7.224,8.261);
\node[gp node center] at (7.224,0.677) {$500$};
\draw[gp path] (8.405,0.985)--(8.405,1.165);
\draw[gp path] (8.405,8.441)--(8.405,8.261);
\node[gp node center] at (8.405,0.677) {$600$};
\draw[gp path] (9.585,0.985)--(9.585,1.165);
\draw[gp path] (9.585,8.441)--(9.585,8.261);
\node[gp node center] at (9.585,0.677) {$700$};
\draw[gp path] (10.766,0.985)--(10.766,1.165);
\draw[gp path] (10.766,8.441)--(10.766,8.261);
\node[gp node center] at (10.766,0.677) {$800$};
\draw[gp path] (11.947,0.985)--(11.947,1.165);
\draw[gp path] (11.947,8.441)--(11.947,8.261);
\node[gp node center] at (11.947,0.677) {$900$};
\draw[gp path] (1.320,8.441)--(1.320,0.985)--(11.947,0.985)--(11.947,8.441)--cycle;
\draw[gp path](6.477,0.986)--(6.465,8.442);
\node[gp node center,rotate=-270] at (0.292,4.713) {$\Delta T$ / s};
\node[gp node center] at (6.633,0.215) {t / s};
\node[gp node right] at (2.424,5.021) {Daten};
\gpcolor{rgb color={0.580,0.000,0.827}}
\gpsetpointsize{4.00}
\gp3point{gp mark 1}{}{(2.028,1.208)}
\gp3point{gp mark 1}{}{(2.737,1.278)}
\gp3point{gp mark 1}{}{(3.445,1.305)}
\gp3point{gp mark 1}{}{(4.154,1.325)}
\gp3point{gp mark 1}{}{(4.862,1.342)}
\gp3point{gp mark 1}{}{(5.571,1.358)}
\gp3point{gp mark 1}{}{(6.279,3.979)}
\gp3point{gp mark 1}{}{(6.988,6.796)}
\gp3point{gp mark 1}{}{(7.696,7.375)}
\gp3point{gp mark 1}{}{(8.405,7.567)}
\gp3point{gp mark 1}{}{(9.113,7.642)}
\gp3point{gp mark 1}{}{(9.822,7.668)}
\gp3point{gp mark 1}{}{(10.530,7.679)}
\gp3point{gp mark 1}{}{(11.239,7.682)}
\gp3point{gp mark 1}{}{(11.947,7.680)}
\gp3point{gp mark 1}{}{(3.066,5.021)}
\gpcolor{color=gp lt color border}
\node[gp node right] at (2.424,4.713) {T1};
\gpcolor{rgb color={0.000,0.620,0.451}}
\draw[gp path] (2.608,4.713)--(3.524,4.713);
\draw[gp path] (2.028,1.234)--(2.129,1.238)--(2.229,1.241)--(2.329,1.245)--(2.429,1.249)%
  --(2.529,1.253)--(2.630,1.257)--(2.730,1.261)--(2.830,1.265)--(2.930,1.268)--(3.030,1.272)%
  --(3.131,1.276)--(3.231,1.280)--(3.331,1.284)--(3.431,1.288)--(3.531,1.292)--(3.631,1.296)%
  --(3.732,1.299)--(3.832,1.303)--(3.932,1.307)--(4.032,1.311)--(4.132,1.315)--(4.233,1.319)%
  --(4.333,1.323)--(4.433,1.326)--(4.533,1.330)--(4.633,1.334)--(4.734,1.338)--(4.834,1.342)%
  --(4.934,1.346)--(5.034,1.350)--(5.134,1.353)--(5.234,1.357)--(5.335,1.361)--(5.435,1.365)%
  --(5.535,1.369)--(5.635,1.373)--(5.735,1.377)--(5.836,1.381)--(5.936,1.384)--(6.036,1.388)%
  --(6.136,1.392)--(6.236,1.396)--(6.337,1.400)--(6.437,1.404)--(6.537,1.408)--(6.637,1.411)%
  --(6.737,1.415)--(6.837,1.419)--(6.938,1.423)--(7.038,1.427)--(7.138,1.431)--(7.238,1.435)%
  --(7.338,1.438)--(7.439,1.442)--(7.539,1.446)--(7.639,1.450)--(7.739,1.454)--(7.839,1.458)%
  --(7.940,1.462)--(8.040,1.465)--(8.140,1.469)--(8.240,1.473)--(8.340,1.477)--(8.440,1.481)%
  --(8.541,1.485)--(8.641,1.489)--(8.741,1.493)--(8.841,1.496)--(8.941,1.500)--(9.042,1.504)%
  --(9.142,1.508)--(9.242,1.512)--(9.342,1.516)--(9.442,1.520)--(9.543,1.523)--(9.643,1.527)%
  --(9.743,1.531)--(9.843,1.535)--(9.943,1.539)--(10.043,1.543)--(10.144,1.547)--(10.244,1.550)%
  --(10.344,1.554)--(10.444,1.558)--(10.544,1.562)--(10.645,1.566)--(10.745,1.570)--(10.845,1.574)%
  --(10.945,1.578)--(11.045,1.581)--(11.146,1.585)--(11.246,1.589)--(11.346,1.593)--(11.446,1.597)%
  --(11.546,1.601)--(11.646,1.605)--(11.747,1.608)--(11.847,1.612)--(11.947,1.616);
\gpcolor{color=gp lt color border}
\node[gp node right] at (2.424,4.405) {T2};
\gpcolor{rgb color={0.337,0.706,0.914}}
\draw[gp path] (2.608,4.405)--(3.524,4.405);
\draw[gp path] (2.028,7.628)--(2.129,7.629)--(2.229,7.630)--(2.329,7.630)--(2.429,7.631)%
  --(2.529,7.631)--(2.630,7.632)--(2.730,7.632)--(2.830,7.633)--(2.930,7.633)--(3.030,7.634)%
  --(3.131,7.634)--(3.231,7.635)--(3.331,7.636)--(3.431,7.636)--(3.531,7.637)--(3.631,7.637)%
  --(3.732,7.638)--(3.832,7.638)--(3.932,7.639)--(4.032,7.639)--(4.132,7.640)--(4.233,7.640)%
  --(4.333,7.641)--(4.433,7.642)--(4.533,7.642)--(4.633,7.643)--(4.734,7.643)--(4.834,7.644)%
  --(4.934,7.644)--(5.034,7.645)--(5.134,7.645)--(5.234,7.646)--(5.335,7.646)--(5.435,7.647)%
  --(5.535,7.648)--(5.635,7.648)--(5.735,7.649)--(5.836,7.649)--(5.936,7.650)--(6.036,7.650)%
  --(6.136,7.651)--(6.236,7.651)--(6.337,7.652)--(6.437,7.652)--(6.537,7.653)--(6.637,7.654)%
  --(6.737,7.654)--(6.837,7.655)--(6.938,7.655)--(7.038,7.656)--(7.138,7.656)--(7.238,7.657)%
  --(7.338,7.657)--(7.439,7.658)--(7.539,7.658)--(7.639,7.659)--(7.739,7.660)--(7.839,7.660)%
  --(7.940,7.661)--(8.040,7.661)--(8.140,7.662)--(8.240,7.662)--(8.340,7.663)--(8.440,7.663)%
  --(8.541,7.664)--(8.641,7.664)--(8.741,7.665)--(8.841,7.666)--(8.941,7.666)--(9.042,7.667)%
  --(9.142,7.667)--(9.242,7.668)--(9.342,7.668)--(9.442,7.669)--(9.543,7.669)--(9.643,7.670)%
  --(9.743,7.670)--(9.843,7.671)--(9.943,7.672)--(10.043,7.672)--(10.144,7.673)--(10.244,7.673)%
  --(10.344,7.674)--(10.444,7.674)--(10.544,7.675)--(10.645,7.675)--(10.745,7.676)--(10.845,7.676)%
  --(10.945,7.677)--(11.045,7.678)--(11.146,7.678)--(11.246,7.679)--(11.346,7.679)--(11.446,7.680)%
  --(11.546,7.680)--(11.646,7.681)--(11.747,7.681)--(11.847,7.682)--(11.947,7.682);
\gpcolor{color=gp lt color border}
\draw[gp path] (1.320,8.441)--(1.320,0.985)--(11.947,0.985)--(11.947,8.441)--cycle;
%% coordinates of the plot area
\gpdefrectangularnode{gp plot 1}{\pgfpoint{1.320cm}{0.985cm}}{\pgfpoint{11.947cm}{8.441cm}}
\end{tikzpicture}
%% gnuplot variables

		\caption{Temperaturdifferenzen der zweiten Benzoesäuremessung}
		\label{abb:benz2}
	\end{subfigure}
	\hfill
	\begin{subfigure}{0.4\textwidth}
		\centering
		\begin{tikzpicture}[gnuplot, scale=0.5, every node/.style={scale=0.5}]
%% generated with GNUPLOT 5.4p5 (Lua 5.4; terminal rev. Jun 2020, script rev. 115)
%% Do 24 Nov 2022 22:18:53 CET
\path (0.000,0.000) rectangle (12.500,8.750);
\gpcolor{color=gp lt color border}
\gpsetlinetype{gp lt border}
\gpsetdashtype{gp dt solid}
\gpsetlinewidth{1.00}
\draw[gp path] (1.320,0.985)--(1.500,0.985);
\draw[gp path] (11.947,0.985)--(11.767,0.985);
\node[gp node right] at (1.136,0.985) {$0$};
\draw[gp path] (1.320,2.050)--(1.500,2.050);
\draw[gp path] (11.947,2.050)--(11.767,2.050);
\node[gp node right] at (1.136,2.050) {$0.5$};
\draw[gp path] (1.320,3.115)--(1.500,3.115);
\draw[gp path] (11.947,3.115)--(11.767,3.115);
\node[gp node right] at (1.136,3.115) {$1$};
\draw[gp path] (1.320,4.180)--(1.500,4.180);
\draw[gp path] (11.947,4.180)--(11.767,4.180);
\node[gp node right] at (1.136,4.180) {$1.5$};
\draw[gp path] (1.320,5.246)--(1.500,5.246);
\draw[gp path] (11.947,5.246)--(11.767,5.246);
\node[gp node right] at (1.136,5.246) {$2$};
\draw[gp path] (1.320,6.311)--(1.500,6.311);
\draw[gp path] (11.947,6.311)--(11.767,6.311);
\node[gp node right] at (1.136,6.311) {$2.5$};
\draw[gp path] (1.320,7.376)--(1.500,7.376);
\draw[gp path] (11.947,7.376)--(11.767,7.376);
\node[gp node right] at (1.136,7.376) {$3$};
\draw[gp path] (1.320,8.441)--(1.500,8.441);
\draw[gp path] (11.947,8.441)--(11.767,8.441);
\node[gp node right] at (1.136,8.441) {$3.5$};
\draw[gp path] (1.320,0.985)--(1.320,1.165);
\draw[gp path] (1.320,8.441)--(1.320,8.261);
\node[gp node center] at (1.320,0.677) {$0$};
\draw[gp path] (2.501,0.985)--(2.501,1.165);
\draw[gp path] (2.501,8.441)--(2.501,8.261);
\node[gp node center] at (2.501,0.677) {$100$};
\draw[gp path] (3.682,0.985)--(3.682,1.165);
\draw[gp path] (3.682,8.441)--(3.682,8.261);
\node[gp node center] at (3.682,0.677) {$200$};
\draw[gp path] (4.862,0.985)--(4.862,1.165);
\draw[gp path] (4.862,8.441)--(4.862,8.261);
\node[gp node center] at (4.862,0.677) {$300$};
\draw[gp path] (6.043,0.985)--(6.043,1.165);
\draw[gp path] (6.043,8.441)--(6.043,8.261);
\node[gp node center] at (6.043,0.677) {$400$};
\draw[gp path] (7.224,0.985)--(7.224,1.165);
\draw[gp path] (7.224,8.441)--(7.224,8.261);
\node[gp node center] at (7.224,0.677) {$500$};
\draw[gp path] (8.405,0.985)--(8.405,1.165);
\draw[gp path] (8.405,8.441)--(8.405,8.261);
\node[gp node center] at (8.405,0.677) {$600$};
\draw[gp path] (9.585,0.985)--(9.585,1.165);
\draw[gp path] (9.585,8.441)--(9.585,8.261);
\node[gp node center] at (9.585,0.677) {$700$};
\draw[gp path] (10.766,0.985)--(10.766,1.165);
\draw[gp path] (10.766,8.441)--(10.766,8.261);
\node[gp node center] at (10.766,0.677) {$800$};
\draw[gp path] (11.947,0.985)--(11.947,1.165);
\draw[gp path] (11.947,8.441)--(11.947,8.261);
\node[gp node center] at (11.947,0.677) {$900$};
\draw[gp path] (1.320,8.441)--(1.320,0.985)--(11.947,0.985)--(11.947,8.441)--cycle;
\draw[gp path](6.506,0.986)--(6.506,8.442);
\node[gp node center,rotate=-270] at (0.292,4.713) {$\Delta T$ / K};
\node[gp node center] at (6.633,0.215) {t / s};
\node[gp node right] at (2.424,5.021) {Daten};
\gpcolor{rgb color={0.580,0.000,0.827}}
\gpsetpointsize{4.00}
\gp3point{gp mark 1}{}{(2.028,1.090)}
\gp3point{gp mark 1}{}{(2.737,1.125)}
\gp3point{gp mark 1}{}{(3.445,1.138)}
\gp3point{gp mark 1}{}{(4.154,1.146)}
\gp3point{gp mark 1}{}{(4.862,1.154)}
\gp3point{gp mark 1}{}{(5.571,1.159)}
\gp3point{gp mark 1}{}{(6.279,2.941)}
\gp3point{gp mark 1}{}{(6.988,6.706)}
\gp3point{gp mark 1}{}{(7.696,7.501)}
\gp3point{gp mark 1}{}{(8.405,7.703)}
\gp3point{gp mark 1}{}{(9.113,7.758)}
\gp3point{gp mark 1}{}{(9.822,7.771)}
\gp3point{gp mark 1}{}{(10.530,7.765)}
\gp3point{gp mark 1}{}{(11.239,7.753)}
\gp3point{gp mark 1}{}{(11.947,7.737)}
\gp3point{gp mark 1}{}{(3.066,5.021)}
\gpcolor{color=gp lt color border}
\node[gp node right] at (2.424,4.713) {T1};
\gpcolor{rgb color={0.000,0.620,0.451}}
\draw[gp path] (2.608,4.713)--(3.524,4.713);
\draw[gp path] (2.028,1.120)--(2.129,1.121)--(2.229,1.122)--(2.329,1.123)--(2.429,1.124)%
  --(2.529,1.126)--(2.630,1.127)--(2.730,1.128)--(2.830,1.129)--(2.930,1.130)--(3.030,1.131)%
  --(3.131,1.132)--(3.231,1.134)--(3.331,1.135)--(3.431,1.136)--(3.531,1.137)--(3.631,1.138)%
  --(3.732,1.139)--(3.832,1.141)--(3.932,1.142)--(4.032,1.143)--(4.132,1.144)--(4.233,1.145)%
  --(4.333,1.146)--(4.433,1.148)--(4.533,1.149)--(4.633,1.150)--(4.734,1.151)--(4.834,1.152)%
  --(4.934,1.153)--(5.034,1.155)--(5.134,1.156)--(5.234,1.157)--(5.335,1.158)--(5.435,1.159)%
  --(5.535,1.160)--(5.635,1.162)--(5.735,1.163)--(5.836,1.164)--(5.936,1.165)--(6.036,1.166)%
  --(6.136,1.167)--(6.236,1.169)--(6.337,1.170)--(6.437,1.171)--(6.537,1.172)--(6.637,1.173)%
  --(6.737,1.174)--(6.837,1.176)--(6.938,1.177)--(7.038,1.178)--(7.138,1.179)--(7.238,1.180)%
  --(7.338,1.181)--(7.439,1.183)--(7.539,1.184)--(7.639,1.185)--(7.739,1.186)--(7.839,1.187)%
  --(7.940,1.188)--(8.040,1.190)--(8.140,1.191)--(8.240,1.192)--(8.340,1.193)--(8.440,1.194)%
  --(8.541,1.195)--(8.641,1.197)--(8.741,1.198)--(8.841,1.199)--(8.941,1.200)--(9.042,1.201)%
  --(9.142,1.202)--(9.242,1.203)--(9.342,1.205)--(9.442,1.206)--(9.543,1.207)--(9.643,1.208)%
  --(9.743,1.209)--(9.843,1.210)--(9.943,1.212)--(10.043,1.213)--(10.144,1.214)--(10.244,1.215)%
  --(10.344,1.216)--(10.444,1.217)--(10.544,1.219)--(10.645,1.220)--(10.745,1.221)--(10.845,1.222)%
  --(10.945,1.223)--(11.045,1.224)--(11.146,1.226)--(11.246,1.227)--(11.346,1.228)--(11.446,1.229)%
  --(11.546,1.230)--(11.646,1.231)--(11.747,1.233)--(11.847,1.234)--(11.947,1.235);
\gpcolor{color=gp lt color border}
\node[gp node right] at (2.424,4.405) {T2};
\gpcolor{rgb color={0.337,0.706,0.914}}
\draw[gp path] (2.608,4.405)--(3.524,4.405);
\draw[gp path] (2.028,7.902)--(2.129,7.900)--(2.229,7.899)--(2.329,7.897)--(2.429,7.895)%
  --(2.529,7.894)--(2.630,7.892)--(2.730,7.890)--(2.830,7.889)--(2.930,7.887)--(3.030,7.885)%
  --(3.131,7.884)--(3.231,7.882)--(3.331,7.881)--(3.431,7.879)--(3.531,7.877)--(3.631,7.876)%
  --(3.732,7.874)--(3.832,7.872)--(3.932,7.871)--(4.032,7.869)--(4.132,7.867)--(4.233,7.866)%
  --(4.333,7.864)--(4.433,7.862)--(4.533,7.861)--(4.633,7.859)--(4.734,7.857)--(4.834,7.856)%
  --(4.934,7.854)--(5.034,7.853)--(5.134,7.851)--(5.234,7.849)--(5.335,7.848)--(5.435,7.846)%
  --(5.535,7.844)--(5.635,7.843)--(5.735,7.841)--(5.836,7.839)--(5.936,7.838)--(6.036,7.836)%
  --(6.136,7.834)--(6.236,7.833)--(6.337,7.831)--(6.437,7.829)--(6.537,7.828)--(6.637,7.826)%
  --(6.737,7.825)--(6.837,7.823)--(6.938,7.821)--(7.038,7.820)--(7.138,7.818)--(7.238,7.816)%
  --(7.338,7.815)--(7.439,7.813)--(7.539,7.811)--(7.639,7.810)--(7.739,7.808)--(7.839,7.806)%
  --(7.940,7.805)--(8.040,7.803)--(8.140,7.801)--(8.240,7.800)--(8.340,7.798)--(8.440,7.797)%
  --(8.541,7.795)--(8.641,7.793)--(8.741,7.792)--(8.841,7.790)--(8.941,7.788)--(9.042,7.787)%
  --(9.142,7.785)--(9.242,7.783)--(9.342,7.782)--(9.442,7.780)--(9.543,7.778)--(9.643,7.777)%
  --(9.743,7.775)--(9.843,7.773)--(9.943,7.772)--(10.043,7.770)--(10.144,7.769)--(10.244,7.767)%
  --(10.344,7.765)--(10.444,7.764)--(10.544,7.762)--(10.645,7.760)--(10.745,7.759)--(10.845,7.757)%
  --(10.945,7.755)--(11.045,7.754)--(11.146,7.752)--(11.246,7.750)--(11.346,7.749)--(11.446,7.747)%
  --(11.546,7.745)--(11.646,7.744)--(11.747,7.742)--(11.847,7.741)--(11.947,7.739);
\gpcolor{color=gp lt color border}
\draw[gp path] (1.320,8.441)--(1.320,0.985)--(11.947,0.985)--(11.947,8.441)--cycle;
%% coordinates of the plot area
\gpdefrectangularnode{gp plot 1}{\pgfpoint{1.320cm}{0.985cm}}{\pgfpoint{11.947cm}{8.441cm}}
\end{tikzpicture}
%% gnuplot variables

		\caption{Temperaturdifferenzen der ersten Probenmessung}
		\label{abb:gummi1}
	\end{subfigure}
	\hspace{5mm}
	\begin{subfigure}{0.4\textwidth}
		\centering
		\begin{tikzpicture}[gnuplot, scale=0.5, every node/.style={scale=0.5}]
%% generated with GNUPLOT 5.4p5 (Lua 5.4; terminal rev. Jun 2020, script rev. 115)
%% Do 24 Nov 2022 22:18:53 CET
\path (0.000,0.000) rectangle (12.500,8.750);
\gpcolor{color=gp lt color border}
\gpsetlinetype{gp lt border}
\gpsetdashtype{gp dt solid}
\gpsetlinewidth{1.00}
\draw[gp path] (1.320,0.985)--(1.500,0.985);
\draw[gp path] (11.947,0.985)--(11.767,0.985);
\node[gp node right] at (1.136,0.985) {$0$};
\draw[gp path] (1.320,2.050)--(1.500,2.050);
\draw[gp path] (11.947,2.050)--(11.767,2.050);
\node[gp node right] at (1.136,2.050) {$0.5$};
\draw[gp path] (1.320,3.115)--(1.500,3.115);
\draw[gp path] (11.947,3.115)--(11.767,3.115);
\node[gp node right] at (1.136,3.115) {$1$};
\draw[gp path] (1.320,4.180)--(1.500,4.180);
\draw[gp path] (11.947,4.180)--(11.767,4.180);
\node[gp node right] at (1.136,4.180) {$1.5$};
\draw[gp path] (1.320,5.246)--(1.500,5.246);
\draw[gp path] (11.947,5.246)--(11.767,5.246);
\node[gp node right] at (1.136,5.246) {$2$};
\draw[gp path] (1.320,6.311)--(1.500,6.311);
\draw[gp path] (11.947,6.311)--(11.767,6.311);
\node[gp node right] at (1.136,6.311) {$2.5$};
\draw[gp path] (1.320,7.376)--(1.500,7.376);
\draw[gp path] (11.947,7.376)--(11.767,7.376);
\node[gp node right] at (1.136,7.376) {$3$};
\draw[gp path] (1.320,8.441)--(1.500,8.441);
\draw[gp path] (11.947,8.441)--(11.767,8.441);
\node[gp node right] at (1.136,8.441) {$3.5$};
\draw[gp path] (1.320,0.985)--(1.320,1.165);
\draw[gp path] (1.320,8.441)--(1.320,8.261);
\node[gp node center] at (1.320,0.677) {$0$};
\draw[gp path] (2.501,0.985)--(2.501,1.165);
\draw[gp path] (2.501,8.441)--(2.501,8.261);
\node[gp node center] at (2.501,0.677) {$100$};
\draw[gp path] (3.682,0.985)--(3.682,1.165);
\draw[gp path] (3.682,8.441)--(3.682,8.261);
\node[gp node center] at (3.682,0.677) {$200$};
\draw[gp path] (4.862,0.985)--(4.862,1.165);
\draw[gp path] (4.862,8.441)--(4.862,8.261);
\node[gp node center] at (4.862,0.677) {$300$};
\draw[gp path] (6.043,0.985)--(6.043,1.165);
\draw[gp path] (6.043,8.441)--(6.043,8.261);
\node[gp node center] at (6.043,0.677) {$400$};
\draw[gp path] (7.224,0.985)--(7.224,1.165);
\draw[gp path] (7.224,8.441)--(7.224,8.261);
\node[gp node center] at (7.224,0.677) {$500$};
\draw[gp path] (8.405,0.985)--(8.405,1.165);
\draw[gp path] (8.405,8.441)--(8.405,8.261);
\node[gp node center] at (8.405,0.677) {$600$};
\draw[gp path] (9.585,0.985)--(9.585,1.165);
\draw[gp path] (9.585,8.441)--(9.585,8.261);
\node[gp node center] at (9.585,0.677) {$700$};
\draw[gp path] (10.766,0.985)--(10.766,1.165);
\draw[gp path] (10.766,8.441)--(10.766,8.261);
\node[gp node center] at (10.766,0.677) {$800$};
\draw[gp path] (11.947,0.985)--(11.947,1.165);
\draw[gp path] (11.947,8.441)--(11.947,8.261);
\node[gp node center] at (11.947,0.677) {$900$};
\draw[gp path] (1.320,8.441)--(1.320,0.985)--(11.947,0.985)--(11.947,8.441)--cycle;
\draw[gp path](6.517,0.986)--(6.517,8.442);
\node[gp node center,rotate=-270] at (0.292,4.713) {$\Delta T$ / s};
\node[gp node center] at (6.633,0.215) {t / s};
\node[gp node right] at (2.424,5.021) {Daten};
\gpcolor{rgb color={0.580,0.000,0.827}}
\gpsetpointsize{4.00}
\gp3point{gp mark 1}{}{(2.028,1.108)}
\gp3point{gp mark 1}{}{(2.737,1.159)}
\gp3point{gp mark 1}{}{(3.445,1.172)}
\gp3point{gp mark 1}{}{(4.154,1.179)}
\gp3point{gp mark 1}{}{(4.862,1.184)}
\gp3point{gp mark 1}{}{(5.571,1.189)}
\gp3point{gp mark 1}{}{(6.279,2.926)}
\gp3point{gp mark 1}{}{(6.988,6.916)}
\gp3point{gp mark 1}{}{(7.696,7.791)}
\gp3point{gp mark 1}{}{(8.405,8.003)}
\gp3point{gp mark 1}{}{(9.113,8.061)}
\gp3point{gp mark 1}{}{(9.822,8.071)}
\gp3point{gp mark 1}{}{(10.530,8.062)}
\gp3point{gp mark 1}{}{(11.239,8.046)}
\gp3point{gp mark 1}{}{(11.947,8.026)}
\gp3point{gp mark 1}{}{(3.066,5.021)}
\gpcolor{color=gp lt color border}
\node[gp node right] at (2.424,4.713) {T1};
\gpcolor{rgb color={0.000,0.620,0.451}}
\draw[gp path] (2.608,4.713)--(3.524,4.713);
\draw[gp path] (2.028,1.155)--(2.129,1.156)--(2.229,1.157)--(2.329,1.158)--(2.429,1.159)%
  --(2.529,1.160)--(2.630,1.161)--(2.730,1.162)--(2.830,1.163)--(2.930,1.164)--(3.030,1.165)%
  --(3.131,1.166)--(3.231,1.167)--(3.331,1.168)--(3.431,1.169)--(3.531,1.170)--(3.631,1.171)%
  --(3.732,1.172)--(3.832,1.173)--(3.932,1.174)--(4.032,1.175)--(4.132,1.176)--(4.233,1.177)%
  --(4.333,1.179)--(4.433,1.180)--(4.533,1.181)--(4.633,1.182)--(4.734,1.183)--(4.834,1.184)%
  --(4.934,1.185)--(5.034,1.186)--(5.134,1.187)--(5.234,1.188)--(5.335,1.189)--(5.435,1.190)%
  --(5.535,1.191)--(5.635,1.192)--(5.735,1.193)--(5.836,1.194)--(5.936,1.195)--(6.036,1.196)%
  --(6.136,1.197)--(6.236,1.198)--(6.337,1.199)--(6.437,1.200)--(6.537,1.201)--(6.637,1.202)%
  --(6.737,1.203)--(6.837,1.204)--(6.938,1.205)--(7.038,1.206)--(7.138,1.207)--(7.238,1.208)%
  --(7.338,1.209)--(7.439,1.210)--(7.539,1.211)--(7.639,1.212)--(7.739,1.213)--(7.839,1.214)%
  --(7.940,1.215)--(8.040,1.216)--(8.140,1.217)--(8.240,1.218)--(8.340,1.219)--(8.440,1.220)%
  --(8.541,1.221)--(8.641,1.223)--(8.741,1.224)--(8.841,1.225)--(8.941,1.226)--(9.042,1.227)%
  --(9.142,1.228)--(9.242,1.229)--(9.342,1.230)--(9.442,1.231)--(9.543,1.232)--(9.643,1.233)%
  --(9.743,1.234)--(9.843,1.235)--(9.943,1.236)--(10.043,1.237)--(10.144,1.238)--(10.244,1.239)%
  --(10.344,1.240)--(10.444,1.241)--(10.544,1.242)--(10.645,1.243)--(10.745,1.244)--(10.845,1.245)%
  --(10.945,1.246)--(11.045,1.247)--(11.146,1.248)--(11.246,1.249)--(11.346,1.250)--(11.446,1.251)%
  --(11.546,1.252)--(11.646,1.253)--(11.747,1.254)--(11.847,1.255)--(11.947,1.256);
\gpcolor{color=gp lt color border}
\node[gp node right] at (2.424,4.405) {T2};
\gpcolor{rgb color={0.337,0.706,0.914}}
\draw[gp path] (2.608,4.405)--(3.524,4.405);
\draw[gp path] (2.028,8.243)--(2.129,8.241)--(2.229,8.238)--(2.329,8.236)--(2.429,8.234)%
  --(2.529,8.232)--(2.630,8.230)--(2.730,8.228)--(2.830,8.225)--(2.930,8.223)--(3.030,8.221)%
  --(3.131,8.219)--(3.231,8.217)--(3.331,8.215)--(3.431,8.212)--(3.531,8.210)--(3.631,8.208)%
  --(3.732,8.206)--(3.832,8.204)--(3.932,8.202)--(4.032,8.199)--(4.132,8.197)--(4.233,8.195)%
  --(4.333,8.193)--(4.433,8.191)--(4.533,8.189)--(4.633,8.186)--(4.734,8.184)--(4.834,8.182)%
  --(4.934,8.180)--(5.034,8.178)--(5.134,8.176)--(5.234,8.173)--(5.335,8.171)--(5.435,8.169)%
  --(5.535,8.167)--(5.635,8.165)--(5.735,8.163)--(5.836,8.160)--(5.936,8.158)--(6.036,8.156)%
  --(6.136,8.154)--(6.236,8.152)--(6.337,8.149)--(6.437,8.147)--(6.537,8.145)--(6.637,8.143)%
  --(6.737,8.141)--(6.837,8.139)--(6.938,8.136)--(7.038,8.134)--(7.138,8.132)--(7.238,8.130)%
  --(7.338,8.128)--(7.439,8.126)--(7.539,8.123)--(7.639,8.121)--(7.739,8.119)--(7.839,8.117)%
  --(7.940,8.115)--(8.040,8.113)--(8.140,8.110)--(8.240,8.108)--(8.340,8.106)--(8.440,8.104)%
  --(8.541,8.102)--(8.641,8.100)--(8.741,8.097)--(8.841,8.095)--(8.941,8.093)--(9.042,8.091)%
  --(9.142,8.089)--(9.242,8.087)--(9.342,8.084)--(9.442,8.082)--(9.543,8.080)--(9.643,8.078)%
  --(9.743,8.076)--(9.843,8.074)--(9.943,8.071)--(10.043,8.069)--(10.144,8.067)--(10.244,8.065)%
  --(10.344,8.063)--(10.444,8.061)--(10.544,8.058)--(10.645,8.056)--(10.745,8.054)--(10.845,8.052)%
  --(10.945,8.050)--(11.045,8.048)--(11.146,8.045)--(11.246,8.043)--(11.346,8.041)--(11.446,8.039)%
  --(11.546,8.037)--(11.646,8.035)--(11.747,8.032)--(11.847,8.030)--(11.947,8.028);
\gpcolor{color=gp lt color border}
\draw[gp path] (1.320,8.441)--(1.320,0.985)--(11.947,0.985)--(11.947,8.441)--cycle;
%% coordinates of the plot area
\gpdefrectangularnode{gp plot 1}{\pgfpoint{1.320cm}{0.985cm}}{\pgfpoint{11.947cm}{8.441cm}}
\end{tikzpicture}
%% gnuplot variables

		\caption{Temperaturdifferenzen der zweiten Probenmessung}
		\label{abb:gummi2}
	\end{subfigure}
	\caption{Messdaten von der kalorimetrischen Bestimmung von Benzoesäure und Gummibärchen sowie die Ausgleichgeraden über die Daten vor und nach der Reaktion}
\end{figure}
Zur Berechnung des Wasserwertes, sind die Temperaturdifferenzen über die Zeit in Abbildungen \ref{abb:benz1} und \ref{abb:benz2} zu sehen.
Es wurde eine lineare Regressionsgerade über die ersten 6 Werte berechnet. 
Eine Zweite wurde über die Letzten vier Werte gelegt.
Über Geogebra wurde die Fläche zwischen einer Vertikalen, den Messpunkten und den Ausgleichsgeraden berechnet.
Die Vertikale wurde so in die Abbildung gelegt, dass beide Flächen gleich groß sind.
Die Schnittpunkte der Vertikalen mit den Ausgleichgeraden gibt die Temperaturdifferenz $\Delta T_\alpha$.
Diese ist befreit von Fehlern, die aus der Wärmestrahlung des Gerätes sowie der Temperaturdifferenz zwischen Wasser und Bombe entsteht.
In Tabelle \ref{tab:1w} sind die berechneten Ausgleichsgeraden, die Vertikalen, Probengewichte sowie die Temperaturdifferenzen dargestellt.

\begin{table}[b]
	\centering
	\begin{tabular}{l|cccc}
\hline
	& Benzoesäure 1      & Benzoesäure 2 	    & Probe 1		   & Probe 2\\
\hline
\hline
m / \unit{\gram}   			& $0,4962$                  & $0,5131                    $ & $2,0421              	$ & $2,1396$\\
T1(t) / \unit{\second}  		& $0,0001t+0,083$            & $0,0001t+0,0475             $ & $0,0001t+0,0594       	$ & $t\cdot 5,66e-5+0,0764$\\
T2(t) / \unit{\second}			& $t\cdot 2,06e-5+1,415$ & $t\cdot 1,38e-5+1,425   $ & $t\cdot -9,11e-5 +3,252  $ & $t\cdot -0,00012+3,414$\\
t / \unit{\second}      		& $437,75$          	    & $436,65                    $ & $439,15              	$ & $440,05$\\
$\Delta T_\alpha$ / \unit{\kelvin} 	& $1,298 $  	       	    & $1,341                     $ & $3,125               	$ & $3,26$\\
\hline
	\end{tabular}
\caption{Ausgangs Daten aus der Messung}
\label{tab:1w}
\end{table}

Gegeben wurden uns die spezifische Verbrennungsenthalpie von Benzoesäure mit \qty{26439}{\joule\per\gram}\cite{skript}.
Und der Verbrennungsenthalpie der Baumwollfäden mit \qty{50}{\joule}.
Daraus ergibt sich nach Gleichung \ref{eq:ww} der Wasserwert.
Beispielhaft ist dies für die erste Benzoesäure-Tablette mit \qty{0,4962}{\gram} und \qty{1,298}{\kelvin} berechnet.
$$ \frac{\qty{0,4962}{\gram}\cdot\qty{26439}{\joule\per\gram}+\qty{50}{\joule}}{\qty{1,298}{\kelvin}} = \qty{10145,63}{\joule\per\kelvin}$$ 
Dies wurde analog für die zweite Tablette durchgeführt.
Wir erhalten folgende Wasserwerte: $W_1= \qty{10145,63}{\joule\per\kelvin}\wedge W_2=\qty{10156,03}{\joule\per\kelvin}$
Daraus folgen ein mittlerer Wasserwert und eine Mittelwertabweichung von: $ \overline{W}=\qty{10152,52}{\joule\per\kelvin}, \Delta W = \qty{3,52}{\joule\per\kelvin}$.
Die Mittelwertabweichung ist in diesem Fall der Gausßschen Fehlerfortpflanzung zu bevorzugen, da diese Größer ist.
Außerdem würde sich nach der Gaußschen Fehlerfortpflanzung (Gleichung \ref{eq:dW}) ein Fehler von $\Delta W_1=\qty{2,04}{\joule\per\kelvin}\wedge\Delta W_2=\qty{1,97}{\joule\per\kelvin}$ ergeben.
Berechnet ist dieser nach:
$$\Delta W_m = \frac{0.0001\ g \cdot 26439 \frac{J}{g}}{1.2976\ K} = 2.0376 \frac{J}{K}$$  
Der Gaußsche Fehler wäre nicht einmal groß genug, den Mittelwert einzuschließen.
Die Temperaturdifferenzen der Proben sind in den Abbildungen \ref{abb:gummi1} und \ref{abb:gummi2} dargestellt.
Die erhaltenen $\Delta T_\alpha$ Werte können nun über Gleichung \ref{eq:ww} in eine Wärmeenergie umgerechnet werden.
Die Verbrennungswärme, geteilt durch die Probenmasse gibt nun die spezifsche Verbrennungswärme. 
Wird nun angenommen, dass die Probe ausschließlich aus Saccharose besteht, so kann über die molare Masse auf die molare Verbrennungsenthalpie von Saccharose geschlossen werden.
Es ergibt sich folgender Zusammenhang:
$$ W=\frac{Q}{\Delta T} \Rightarrow W\Delta T = Q \Rightarrow \frac{W\Delta T}{m} = Q_m \Rightarrow \frac{WM\Delta T}{m} = Q_M$$
Für die erste Probenmessung mit $M=\qty{342,3}{\gram\per\mole}, \Delta T=\qty{3,124}{\kelvin}$ und $m=\qty{2,0421}{\gram}$, folgt eine molare Verbrennungsenthalpie von:
$$ Q_{M1} = \frac{\qty{342,3}{\gram\per\mole}\cdot \qty{3,124}{\kelvin}\cdot\qty{10152,52}{\joule\per\kelvin}}{\qty{2,0421}{\gram}} = \qty{5317524,91}{\joule\per\mole}$$
Aus dem zweiten Experiment ergibt sich eine molare Verbrennungsenthalpie von \\$Q_{M2}=\qty{-5294899,02}{\joule\per\mole}$.
Der Fehler kann wieder über die Mittelwertabweichung oder der Gaußschen Fehlerfortpflanzung berechnet werden.
Ermittelt wurde eine Mittelwertabweichung von $\overline{\Delta Q} = \qty{11312,94}{\joule\per\mole}$.
Und ein gaußscher Fehler von $\Delta Q_{M1} = \qty{1842,21}{\joule\per\mole} \wedge \Delta Q_{M2}=\qty{1834,37}{\joule\per\mole}$.
Beispielhaft ist der gaußsche Fehler mit  $\Delta T=\qty{3,1247}{\kelvin}, m=\qty{2,0421}{\gram}, W=\qty{10152
52}{\joule\per\kelvin}$und$ \Delta W=\qty{3,517}{\joule\per\kelvin}$ wie folgt berechnet worden:
\begin{align*}
\Delta Q_M &=\qty{3,1247}{\kelvin}\cdot \qty{342,294}{\gram\per\mole}\cdot \sqrt{\left( - \frac{0,0001\ g \cdot \qty{10152,52}{\joule\per\kelvin}}{\qty{2,0421}{\gram}^2}\right)^2 + \left( \frac{\qty{3,517}{\joule\per\kelvin}}{\qty{2,0421}{\gram}}\right)^2}\\
\Delta Q_M &= \qty{1860,53}{\joule\per\mole}
\end{align*}
Die Reaktionsenergie wird nun über die Gleichung \ref{eq:UR} bestimmt.
Die Reaktionsenthalpie ergibt sich nun aus:
$$ \Delta H_R = \Delta U_R + \Delta\nu RT$$
In unserem Fall errechnet sich $\Delta\nu$ für Saccharose ($C_{12}H_{22}O_{11}$) nach:
$$\Delta\nu = \frac{11+0}{2}-\frac{22}{4} = 0$$
Damit ist die Verbrennungsenthalpie gleich der Reaktionsenergie.
Die Standartbildungsenthalpie ist außerdem gleich:
$$\Delta H_B = \nu_i\Delta H_{R,i}-\Delta H_R$$
In unserem Fall ergibt sich mit $\Delta H_{RH} = \qty{-285,98}{\kilo\joule\per\mole}$ und  $\Delta H_{CR} = \qty{-393,42}{\kilo\joule\per\mole}$ eine Standartbildungsenergie von:
$$\Delta H_B=22\cdot\qty{-285,98}{\kilo\joule\per\mole} + 12\cdot \Delta H_{CR} = \qty{-393,42}{\kilo\joule\per\mole} - \qty{11093,94}{\kilo\joule\per\mole}$$
Damit hat Saccharose eine Standartbildungsenergie von: $\Delta H_B=\qty{-33200,48}{\kilo\joule\per\mole}$
Eine Fehlerrechnung ist hier überflüssig, da hier kein zusätzlicher Fehler entsteht.

Nach der Literatur \cite{atkins} liegt die molare Verbrennungsenthalpie bei \qty{-5645000}{\joule\per\mole}.
Nach beiden Fehlerermittlungsmethoden liegt der Literaturwert nicht im Fehlerbereich. 
Somit folgt, dass unsere Messung stärker fehlerbehaftet ist als vorher angenommen.
Die erhaltenen  Ergebnisse können aus der Tabelle \ref{tab:erg} entnommen werden.
\begin{table}[h!]
\centering
\begin{tabular}{l|cc}
	& Messung 1 & Messung 2\\
\hline
$H_B$  /\unit{\joule\per\mole} 		& \multicolumn{2}{c}{-33200483,57}\\\\
$U_R$  /\unit{\joule\per\mole} 		& 10858917,61    & 11328965,95\\
$U_R$  /\unit{\joule\per\mole} 		& \multicolumn{2}{c}{11093941,78}\\\\
$dQ$  /\unit{\joule\per\mole} 		& 1860,53        & 1850,99\\
$dQ$  /\unit{\joule\per\mole} 		& \multicolumn{2}{c}{11312,94}\\\\
$Q$   /\unit{\joule\per\mole} 		& 5317524,91     & 5294899,02\\\\
$dW$  /\unit{\joule\per\mole\per\kelvin}	& 2,04           & 1,97\\
$dW$  /\unit{\joule\per\mole\per\kelvin} 	& \multicolumn{2}{c}{3,52}\\\\
$W$   /\unit{\joule\per\mole\per\kelvin}	& 10149          & 10156,03\\
$W$   /\unit{\joule\per\mole\per\kelvin}	& \multicolumn{2}{c}{10152,52}\\
\end{tabular}
\caption{Ergebnisse der kalorimetrischen Messung von Benzoesäure und Saccharose}
\label{tab:erg}
\end{table}
