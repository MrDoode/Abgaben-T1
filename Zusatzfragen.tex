\section{Zusatzfragen}
	\begin{enumerate}
		\item \textbf{ Was versteht man unter adiabatischer und isothermer Kalorimetrie? In welche Gruppe gehört der  vorstehende Versuch?} \\
		Bei der adiabatischen Kalorimetrie findet kein Wärmeaustausch statt und bei der isothermen Kalorimetrie bleibt die Temperatur konstant. 
		Der durchgeführte Versuch gehört zur adiabatischen Kalorimetrie.	
		\item \textbf{ Was sind atomare Bildungsenthalpien?}\\
		Atomare Bildungsenthalpien sind die Bildungsenthalpien von den einzelnen Atomen. Auch Atomisierungsenergie genannt.
		\item \textbf{ Warum muss nach der DIN-Vorschrift die Bombe vor der Verbrennung mit 5\ ml Wasser gefüllt werden? Welcher Fehler kann außerdem dadurch ausgeschlossen werden und wie groß ist er?}\\
		Die Zugabe von Wasser hilft bei der Kondensation von Wasser und verhindert das frühzeitige Stoppen der Reaktion. Wenn die Verbrennung vorzeitig abbricht, wäre der Fehler sehr groß.   
		\item  \textbf{ Wie groß ist der Fehler, wenn man bei der verbrannten Substanz näherungsweise Verbrennungsenthalpie und -energie gleichsetzt?}\\
		Wenn wir eine isobare Messung durchführen, ist der Fehler vernachlässigbar.
		$dH = dQ + pdV$
		Bei einer nicht isobaren Messung ist liegt ein großer Fehler vor.
		$dH = dQ + Vdp$
		Gehen wir in unserem Fall von einem Isobaren Prozess aus, ist der Fehler Vernachlässigbar, da $dp=0$ ist und damit der gesamt zweite Term null ist.
		In einem nicht isobaren System kann eine Abschätzung des Fehlers durch errechnen der Druckdifferenz nach dem Idealen Gasgesetz gemacht werden.
		Für die Saccharose Verbrennung liegt nach Reaktionsgleichung \ref{eq:rkt} folgende Reaktionsgleichung vor:
		$$ C_{12}H_{22}O_{11} + 12O_2 \rightarrow 12CO_2 + 11H_2O $$
		Daraus ergibt sich ein $\Delta n$ von \qty{10}{\mole}
		Das Volumen des reaktionsgefäßes ist irrelevant, da sich durch das Multiplizieren von der Druckänderung mit dem Volumen dieses herauskürzt, wenn die Druckänderung über das Ideale Gasgesetz errechnet wird.
		Weiter wird eine Temperatur von \qty{300}{\kelvin} angenommen.
		So folgt:
		\begin{align*}
			Vdp &= V\cdot\frac{\Delta nRT}{V}\\
			&= \qty{10}{\mole}\cdot\qty{8,31446}{\joule\per\mole\per\kelvin}\cdot\qty{300}{\kelvin} \\
			&= \qty{24943,38}{\joule}
		\end{align*}	
		$$ $$
	\end{enumerate}
