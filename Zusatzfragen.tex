\section{Zusatzfragen}
	\begin{enumerate}
		\item \textbf{ Was versteht man unter adiabatischer und isothermer Kalorimetrie? In welche Gruppe gehört der  vorstehende Versuch?} \\
		Bei der adiabatischen Kalorimetrie findet kein Wärmeaustausch statt und bei der isothermer Kalorimetrie bleibt die Temperatur konstant. 
		Der durchgeführte versuch gehört zur adiabatischer Kalorimetrie. 
	
		\item \textbf{ Was sind atomare Bildungsenthalpien?}\\
		Atomare Bildungsenthalpien sind die Bildungsenthalpien der einzelnen Atomen. Auch atomisierungsenergie genannt.
		\item \textbf{ Warum muss nach der DIN-Vorschrift die Bombe vor der Verbrennung mit 5\ ml Wasser gefüllt werden? Welcher Fehler kann außerdem dadurch ausgeschlossen werden und wie groß ist er?}\\
		Die Zugabe von Wasser hilft bei der Kondensation von Wasser und verhindert das frühzeitige stoppen der Reaktion. Wenn die Verbrennung vorzeitig abbricht wäre der Fehler sehr groß.   
		\item  \textbf{ Wie groß ist der Fehler, wenn man bei der verbrannten Substanz näherungsweise Verbrennungsenthalpie und -energie gleichsetzt?}\\
		Wenn wir eine isobare Messung durchführen ist der Fehler vernachlässigbar.
		$dH = dQ \cdot Vdp$
		Bei einer nicht isobaren Messung ist es ein großer Fehler.
	\end{enumerate}
