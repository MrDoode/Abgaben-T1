\section{Theoretischer Hintergrund}
	Zur Bestimmung der entstehenden Verbrennungswärme benutzt man die kalorimetrische Bombe. 
	Die Substanz wird unter Sauerstoffdruck verbrannt. 
	Mit der frei werdenden Verbrennungswärme $Q_V$ kann durch Umrechnungen die frei werdende Reaktionsenergie ($U_R$) 
	\begin{equation}
		m\cdot Q_V = n\Delta U_R
		\label{eq:UR}
	\end{equation}
	 ermittel werden.
	Die folgende Reaktionsgleichung zeigt den allgemeinen Fall der Verbrennung einer Substanz:
	\begin{equation}
	 	C_aH_bO_cN_d + \left( a + \frac{b}{4} - \frac{c}{2} \right) O_{2(g)} \rightarrow aCO_{2(g)} + \frac{b}{2} H_2O_{(1)} + \frac{d}{2} N_{2(g)} 
		\label{eq:rkt}
	\end{equation}
	 Die Molzahldifferenz kann mit 
	\begin{equation}
	 	\Delta \nu = \frac{c + d}{2} - \frac{b}{4}
		\label{eq:nu}
	\end{equation}  
	berechnet werden. 
	Der Wasserwert gibt  die Wärmekapazität des gesamten Systems wieder. 
	Die Berechnung des Wasserwertes erfolgt mit der Formel:
	\begin{equation}
		W = \frac{Q}{\Delta T}  \wedge Q = \Delta T \cdot W
		\label{eq:ww}
	\end{equation}
	Der Fehler des Wasserwertes kann über die gaußsche Fehlerfortpflanzung nach Gleichung \ref{eq:dW} oder über die Ermittlung des Mittelwertes bestimmt werden.
	\begin{equation}
		\Delta W_m = \sqrt{\left(\frac{\delta W_m}{\delta m}\cdot \Delta m\right)^2} = \frac{\Delta m \cdot Q}{\Delta T}\\	
		\label{eq:dW}
	\end{equation}
	Der Fehler von $Q$, bzw. $Q_M$ wird analog über Gleichung \ref{eq:dQ} bestimmt.
	\begin{equation}
		\Delta Q_M = \sqrt{\left( \frac{\delta Q_M}{\delta m}\cdot \Delta m \right)^2 + \left( \frac{\delta Q_M}{\delta W}\cdot \Delta W \right)^2}
     = \Delta T \cdot M \cdot \sqrt{\left( - \frac{\Delta m \cdot W}{m^2}\right)^2 + \left( \frac{\Delta W}{m}\right)^2}				
		\label{eq:dQ}
	\end{equation}
