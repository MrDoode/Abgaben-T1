\section{Theoretischer Hinterung}
	Zur Bestimmung der entstehenden Verbrennungswärme benutzt man die kalorimetrische Bombe. 
	Die Substanz wird unter Sauerstoffdruck verbrannt. 
	Mit der freiwerdende Verbrennungswärme $Q_V$ kann durch Umrechnungen die freiwerdende Reaktionsenergie $\Delta U_R$ durch $m\cdot Q_V = n\Delta U_R$ ermittel werden.
	Die folgende Reaktionsgleichung zeigt den allgemeinen Fall der Verbrennung einer Substanz:
	 $	C_aH_bO_cN_d + \left( a + \frac{b}{4} - \frac{c}{2} \right) O_{2(g)} \rightarrow aCO_{2(g)} + \frac{b}{2} H_2O_{(1)} + \frac{d}{2} N_{2(g)} $ 
	 Die Molzahldifferenz kann mit 
	 $	\Delta \nu = \frac{c + d}{2} - \frac{b}{4} $ berechnet werden. 
	Der Wasserwert gibt  die Wärmekapazität des gesamten Systems wieder. 
	Die Berechnung des Wasserwertes erfolgt mit der Formel:
	 $	W = \frac{Q}{\Delta T} $
	\section{Rechnung}
	\begin{eqnarray}
		C_aH_bO_cN_d + \left( a + \frac{b}{4} - \frac{c}{2} \right) O_{2(g)} \rightarrow aCO_{2(g)} + \frac{b}{2} H_2O_{(1)} + \frac{d}{2} N_{2(g)}\\	
		\Delta \nu = \frac{c + d}{2} - \frac{b}{4}\\
		W = \frac{Q}{\Delta T}\\
		Q = \Delta T \cdot W\\
		\Delta W = \frac{1}{\Delta T}\cdot Q\\
		\Delta W = \frac{1}{1.2976\ K}\cdot 26439 \frac{J}{g}\cdot 0.0001\ g = 2.0376 \frac{J}{K}
	\end{eqnarray}
